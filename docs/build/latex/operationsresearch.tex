%% Generated by Sphinx.
\def\sphinxdocclass{report}
\documentclass[letterpaper,10pt,english]{sphinxmanual}
\ifdefined\pdfpxdimen
   \let\sphinxpxdimen\pdfpxdimen\else\newdimen\sphinxpxdimen
\fi \sphinxpxdimen=.75bp\relax
\ifdefined\pdfimageresolution
    \pdfimageresolution= \numexpr \dimexpr1in\relax/\sphinxpxdimen\relax
\fi
%% let collapsible pdf bookmarks panel have high depth per default
\PassOptionsToPackage{bookmarksdepth=5}{hyperref}

\PassOptionsToPackage{warn}{textcomp}
\usepackage[utf8]{inputenc}
\ifdefined\DeclareUnicodeCharacter
% support both utf8 and utf8x syntaxes
  \ifdefined\DeclareUnicodeCharacterAsOptional
    \def\sphinxDUC#1{\DeclareUnicodeCharacter{"#1}}
  \else
    \let\sphinxDUC\DeclareUnicodeCharacter
  \fi
  \sphinxDUC{00A0}{\nobreakspace}
  \sphinxDUC{2500}{\sphinxunichar{2500}}
  \sphinxDUC{2502}{\sphinxunichar{2502}}
  \sphinxDUC{2514}{\sphinxunichar{2514}}
  \sphinxDUC{251C}{\sphinxunichar{251C}}
  \sphinxDUC{2572}{\textbackslash}
\fi
\usepackage{cmap}
\usepackage[T1]{fontenc}
\usepackage{amsmath,amssymb,amstext}
\usepackage{babel}



\usepackage{tgtermes}
\usepackage{tgheros}
\renewcommand{\ttdefault}{txtt}



\usepackage[Bjarne]{fncychap}
\usepackage{sphinx}

\fvset{fontsize=auto}
\usepackage{geometry}


% Include hyperref last.
\usepackage{hyperref}
% Fix anchor placement for figures with captions.
\usepackage{hypcap}% it must be loaded after hyperref.
% Set up styles of URL: it should be placed after hyperref.
\urlstyle{same}

\addto\captionsenglish{\renewcommand{\contentsname}{Table of Contents}}

\usepackage{sphinxmessages}
\setcounter{tocdepth}{4}
\setcounter{secnumdepth}{4}


\title{Operations Research Documentation}
\date{Nov 05, 2021}
\release{1.0}
\author{Qiuyi Hong}
\newcommand{\sphinxlogo}{\vbox{}}
\renewcommand{\releasename}{Release}
\makeindex
\begin{document}

\ifdefined\shorthandoff
  \ifnum\catcode`\=\string=\active\shorthandoff{=}\fi
  \ifnum\catcode`\"=\active\shorthandoff{"}\fi
\fi

\pagestyle{empty}
\sphinxmaketitle
\pagestyle{plain}
\sphinxtableofcontents
\pagestyle{normal}
\phantomsection\label{\detokenize{index::doc}}



\chapter{Linear Programming}
\label{\detokenize{operationsResearch/LP:linear-programming}}\label{\detokenize{operationsResearch/LP::doc}}
\begin{sphinxShadowBox}
\sphinxstyletopictitle{Contents}
\begin{itemize}
\item {} 
\sphinxAtStartPar
\phantomsection\label{\detokenize{operationsResearch/LP:id1}}{\hyperref[\detokenize{operationsResearch/LP:linear-programming}]{\sphinxcrossref{Linear Programming}}}
\begin{itemize}
\item {} 
\sphinxAtStartPar
\phantomsection\label{\detokenize{operationsResearch/LP:id2}}{\hyperref[\detokenize{operationsResearch/LP:lp-model}]{\sphinxcrossref{LP Model}}}

\item {} 
\sphinxAtStartPar
\phantomsection\label{\detokenize{operationsResearch/LP:id3}}{\hyperref[\detokenize{operationsResearch/LP:solution-methods}]{\sphinxcrossref{Solution Methods}}}
\begin{itemize}
\item {} 
\sphinxAtStartPar
\phantomsection\label{\detokenize{operationsResearch/LP:id4}}{\hyperref[\detokenize{operationsResearch/LP:the-simplex-method}]{\sphinxcrossref{The Simplex Method}}}
\begin{itemize}
\item {} 
\sphinxAtStartPar
\phantomsection\label{\detokenize{operationsResearch/LP:id5}}{\hyperref[\detokenize{operationsResearch/LP:the-matrix-way-for-the-simplex-method}]{\sphinxcrossref{The matrix way for the simplex method}}}

\end{itemize}

\end{itemize}

\item {} 
\sphinxAtStartPar
\phantomsection\label{\detokenize{operationsResearch/LP:id6}}{\hyperref[\detokenize{operationsResearch/LP:applications}]{\sphinxcrossref{Applications}}}

\end{itemize}

\end{itemize}
\end{sphinxShadowBox}


\section{LP Model}
\label{\detokenize{operationsResearch/LP:lp-model}}

\section{Solution Methods}
\label{\detokenize{operationsResearch/LP:solution-methods}}

\subsection{The Simplex Method}
\label{\detokenize{operationsResearch/LP:the-simplex-method}}

\subsubsection{The matrix way for the simplex method}
\label{\detokenize{operationsResearch/LP:the-matrix-way-for-the-simplex-method}}
\sphinxAtStartPar
The easiest way of doing this is to use the \sphinxstylestrong{matrix representation} for the simplex method.

\sphinxAtStartPar
Consider a standard form linear program:
\begin{equation*}
\begin{split}\max_{x} c^{T}x\end{split}
\end{equation*}
\sphinxAtStartPar
subject to:
\begin{align*}\!\begin{aligned}
Ax=b\\
x \geq 0\\
\end{aligned}\end{align*}
\sphinxAtStartPar
By dividing \(x\) to \(x_{B}\) and \(x_{N}\), the basic and non\sphinxhyphen{}basic variables,
we have:
\begin{equation*}
\begin{split}\max_{x_{B},x_{N}} c_{B}^{T}x_{B} + c_{N}^{T}x_{N}\end{split}
\end{equation*}
\sphinxAtStartPar
subject to:
\begin{align*}\!\begin{aligned}
& A_{B}x_{B} + A_{N}x_{N} = b\\
& x_{B},x_{N} \geq 0\\
\end{aligned}\end{align*}
\sphinxAtStartPar
where \(c^{T} = [c_{B}^{T},c_{N}^{T}]\), \(A=[A_{B}, A_{N}]\).

\sphinxAtStartPar
After rearrange the optimization problem, we obtain:
\begin{equation*}
\begin{split}\max_{x_{N}} c_{B}^{T}A_{B}^{-1}b - (c_{B}^{T}A_{B}^{-1}A_{N} - c_{N}^{T})x_{N}\end{split}
\end{equation*}
\sphinxAtStartPar
subject to:
\begin{align*}\!\begin{aligned}
& I x_{B}+A_{B}^{-1}A_{N}x_{N} = A_{B}^{-1}b\\
& x_{B}, x_{N} \geq 0\\
\end{aligned}\end{align*}
\sphinxAtStartPar
Let’s ignore the sign constraints and let \(z\) be the objective value. We then have:
\begin{align*}\!\begin{aligned}
& z \quad + (c_{B}^{T}A_{B}^{-1}A_{N} - c_{N}^{T})x_{N} = c_{B}^{T}A_{B}^{-1}b\\
& \quad I x_{B}+ \quad A_{B}^{-1}A_{N}x_{N} = A_{B}^{-1}b\\
\end{aligned}\end{align*}
\sphinxAtStartPar
Therefore, given \sphinxstylestrong{any valid choice of B and N} (the index sets of basic and non\sphinxhyphen{}basic variables),
we may use the following table to calculate the simplex tableau:


\begin{savenotes}\sphinxattablestart
\centering
\begin{tabulary}{\linewidth}[t]{|T|T|T|}
\hline
\sphinxstyletheadfamily 
\sphinxAtStartPar
0
&\sphinxstyletheadfamily 
\sphinxAtStartPar
\(c_{B}^{T}A_{B}^{-1}A_{N} - c_{N}^{T}\)
&\sphinxstyletheadfamily 
\sphinxAtStartPar
\(c_{B}^{T}A_{B}^{-1}b\)
\\
\hline
\sphinxAtStartPar
\(I\)
&
\sphinxAtStartPar
\(A_{B}^{-1}A_{N}\)
&
\sphinxAtStartPar
\(A_{B}^{-1}b\)
\\
\hline
\sphinxAtStartPar
basic
&
\sphinxAtStartPar
non\sphinxhyphen{}basic
&
\sphinxAtStartPar
RHS
\\
\hline
\end{tabulary}
\par
\sphinxattableend\end{savenotes}


\section{Applications}
\label{\detokenize{operationsResearch/LP:applications}}

\chapter{Integer Programming}
\label{\detokenize{operationsResearch/IP:integer-programming}}\label{\detokenize{operationsResearch/IP::doc}}
\begin{sphinxShadowBox}
\sphinxstyletopictitle{Contents}
\begin{itemize}
\item {} 
\sphinxAtStartPar
\phantomsection\label{\detokenize{operationsResearch/IP:id4}}{\hyperref[\detokenize{operationsResearch/IP:integer-programming}]{\sphinxcrossref{Integer Programming}}}
\begin{itemize}
\item {} 
\sphinxAtStartPar
\phantomsection\label{\detokenize{operationsResearch/IP:id5}}{\hyperref[\detokenize{operationsResearch/IP:ip-model}]{\sphinxcrossref{IP Model}}}

\item {} 
\sphinxAtStartPar
\phantomsection\label{\detokenize{operationsResearch/IP:id6}}{\hyperref[\detokenize{operationsResearch/IP:solution-methods}]{\sphinxcrossref{Solution Methods}}}
\begin{itemize}
\item {} 
\sphinxAtStartPar
\phantomsection\label{\detokenize{operationsResearch/IP:id7}}{\hyperref[\detokenize{operationsResearch/IP:the-branch-and-bound-algorithm}]{\sphinxcrossref{The Branch and Bound Algorithm}}}

\end{itemize}

\item {} 
\sphinxAtStartPar
\phantomsection\label{\detokenize{operationsResearch/IP:id8}}{\hyperref[\detokenize{operationsResearch/IP:applications}]{\sphinxcrossref{Applications}}}
\begin{itemize}
\item {} 
\sphinxAtStartPar
\phantomsection\label{\detokenize{operationsResearch/IP:id9}}{\hyperref[\detokenize{operationsResearch/IP:machine-scheduling-problem}]{\sphinxcrossref{Machine Scheduling Problem}}}

\item {} 
\sphinxAtStartPar
\phantomsection\label{\detokenize{operationsResearch/IP:id10}}{\hyperref[\detokenize{operationsResearch/IP:facility-location-problem}]{\sphinxcrossref{Facility Location Problem}}}

\end{itemize}

\end{itemize}

\end{itemize}
\end{sphinxShadowBox}


\section{IP Model}
\label{\detokenize{operationsResearch/IP:ip-model}}
\sphinxAtStartPar
Hello World


\section{Solution Methods}
\label{\detokenize{operationsResearch/IP:solution-methods}}

\subsection{The Branch and Bound Algorithm}
\label{\detokenize{operationsResearch/IP:the-branch-and-bound-algorithm}}

\section{Applications}
\label{\detokenize{operationsResearch/IP:applications}}

\subsection{Machine Scheduling Problem}
\label{\detokenize{operationsResearch/IP:machine-scheduling-problem}}
\sphinxAtStartPar
Fifteen jobs, each with its processing time, should be scheduled on three machines.
If two jobs cannot be scheduled on the same machine, they are called conflicting jobs.
Table 1 lists the job IDs, processing times, and sets of conflicting jobs.
For example, we cannot schedule any pair of jobs out of jobs 2, 5, 8 on the same machine.
Note that a job may have no conflicting jobs.


\begin{savenotes}\sphinxattablestart
\centering
\sphinxcapstartof{table}
\sphinxthecaptionisattop
\sphinxcaption{Table 1: Data}\label{\detokenize{operationsResearch/IP:id1}}
\sphinxaftertopcaption
\begin{tabulary}{\linewidth}[t]{|T|T|T|}
\hline
\sphinxstyletheadfamily 
\sphinxAtStartPar
Job
&\sphinxstyletheadfamily 
\sphinxAtStartPar
Processing time
&\sphinxstyletheadfamily 
\sphinxAtStartPar
Conflicting jobs
\\
\hline
\sphinxAtStartPar
1
&
\sphinxAtStartPar
7
&
\sphinxAtStartPar
None
\\
\hline
\sphinxAtStartPar
2
&
\sphinxAtStartPar
4
&
\sphinxAtStartPar
5,8
\\
\hline
\sphinxAtStartPar
3
&
\sphinxAtStartPar
6
&
\sphinxAtStartPar
None
\\
\hline
\sphinxAtStartPar
4
&
\sphinxAtStartPar
9
&
\sphinxAtStartPar
None
\\
\hline
\sphinxAtStartPar
5
&
\sphinxAtStartPar
12
&
\sphinxAtStartPar
2,8
\\
\hline
\sphinxAtStartPar
6
&
\sphinxAtStartPar
8
&
\sphinxAtStartPar
9
\\
\hline
\sphinxAtStartPar
7
&
\sphinxAtStartPar
10
&
\sphinxAtStartPar
10
\\
\hline
\sphinxAtStartPar
8
&
\sphinxAtStartPar
11
&
\sphinxAtStartPar
2,5
\\
\hline
\sphinxAtStartPar
9
&
\sphinxAtStartPar
8
&
\sphinxAtStartPar
6
\\
\hline
\sphinxAtStartPar
10
&
\sphinxAtStartPar
7
&
\sphinxAtStartPar
7
\\
\hline
\sphinxAtStartPar
11
&
\sphinxAtStartPar
6
&
\sphinxAtStartPar
15
\\
\hline
\sphinxAtStartPar
12
&
\sphinxAtStartPar
8
&
\sphinxAtStartPar
None
\\
\hline
\sphinxAtStartPar
13
&
\sphinxAtStartPar
15
&
\sphinxAtStartPar
None
\\
\hline
\sphinxAtStartPar
14
&
\sphinxAtStartPar
14
&
\sphinxAtStartPar
None
\\
\hline
\sphinxAtStartPar
15
&
\sphinxAtStartPar
3
&
\sphinxAtStartPar
11
\\
\hline
\end{tabulary}
\par
\sphinxattableend\end{savenotes}

\sphinxAtStartPar
We want to schedule the jobs to minimize makespan. For example, we may schedule jobs 1, 4, 7, 8,
and 13 to machine 1, jobs 2, 6, 10, 11, and 14 to machine 2, and jobs 3, 5, 9, 12, and 15 to machine 3.
The total processing times on the three machines are 52, 39, and 37, respectively.
The makespan is thus 52. While this is a feasible schedule, this may or may not be an optimal schedule.
When we try to improve the schedule, be careful about conflicting jobs. For example, we cannot exchange
jobs 8 and 11 (even though this reduces the makespan) because that will result in machine 2 processing
conflicting jobs 2 and 8, which is infeasible.

\sphinxAtStartPar
Formulate a linear integer program that generates a feasible schedule to minimize makespan.
Then write a computer program (e.g., using Python to invoke Gurobi Optimizer) to solve this instance
and obtain an optimal schedule. Write down the minimized makespan (i.e. the objective value of
an optimal solution).
\begin{equation*}
\begin{split}\min_{X_{i,j}, w} w\end{split}
\end{equation*}
\sphinxAtStartPar
subject to:
\begin{equation*}
\begin{split}w \geq \sum_{j \in \mathcal{J}} t_{j} X_{i,j} , \forall i \in \mathcal{I} \\
\sum_{i \in \mathcal{I}} X_{i,j} = 1 , \forall j \in \mathcal{J} \\
\sum_{j \in \Omega_{1,2,3,4}} X_{i,j} \leq 1 , \forall i \in \mathcal{I}\end{split}
\end{equation*}
\sphinxAtStartPar
where \(i \in \mathcal{I}\) denotes machine index. \(j \in \mathcal{J}\) denotes job index.

\sphinxAtStartPar
Conflicting job sets are \(\left\{\begin{matrix}
\Omega_{1} & \{ 2,5,8 \}\\
\Omega_{2} & \{ 6,9 \}\\
\Omega_{3} & \{ 7,10 \}\\
\Omega_{4} & \{ 11,15 \}
\end{matrix}\right.\)

\sphinxAtStartPar
Decision variable \(X_{i,j} = \left\{\begin{matrix}
1 & \text{if job } j \text{ is allocated to machine } i\\
0 & \text{otherwise}
\end{matrix}\right.\)

\sphinxAtStartPar
Decision variable \(w\) is a trivial positive value.

\begin{sphinxVerbatim}[commandchars=\\\{\}]
\PYG{k+kn}{import} \PYG{n+nn}{numpy} \PYG{k}{as} \PYG{n+nn}{np}
\PYG{k+kn}{import} \PYG{n+nn}{cvxpy} \PYG{k}{as} \PYG{n+nn}{cp}

\PYG{n}{X} \PYG{o}{=} \PYG{n}{cp}\PYG{o}{.}\PYG{n}{Variable}\PYG{p}{(}\PYG{n}{shape}\PYG{o}{=}\PYG{p}{(}\PYG{l+m+mi}{3}\PYG{p}{,}\PYG{l+m+mi}{15}\PYG{p}{)}\PYG{p}{,}\PYG{n}{boolean}\PYG{o}{=}\PYG{k+kc}{True}\PYG{p}{)}
\PYG{n}{w} \PYG{o}{=} \PYG{n}{cp}\PYG{o}{.}\PYG{n}{Variable}\PYG{p}{(}\PYG{p}{)}

\PYG{n}{t} \PYG{o}{=} \PYG{n}{np}\PYG{o}{.}\PYG{n}{array}\PYG{p}{(}\PYG{p}{[}\PYG{p}{[}\PYG{l+m+mi}{7}\PYG{p}{,}\PYG{l+m+mi}{4}\PYG{p}{,}\PYG{l+m+mi}{6}\PYG{p}{,}\PYG{l+m+mi}{9}\PYG{p}{,}\PYG{l+m+mi}{12}\PYG{p}{,}\PYG{l+m+mi}{8}\PYG{p}{,}\PYG{l+m+mi}{10}\PYG{p}{,}\PYG{l+m+mi}{11}\PYG{p}{,}\PYG{l+m+mi}{8}\PYG{p}{,}\PYG{l+m+mi}{7}\PYG{p}{,}\PYG{l+m+mi}{6}\PYG{p}{,}\PYG{l+m+mi}{8}\PYG{p}{,}\PYG{l+m+mi}{15}\PYG{p}{,}\PYG{l+m+mi}{14}\PYG{p}{,}\PYG{l+m+mi}{3}\PYG{p}{]}\PYG{p}{]}\PYG{p}{)}\PYG{o}{.}\PYG{n}{T}

\PYG{n}{obj} \PYG{o}{=} \PYG{n}{cp}\PYG{o}{.}\PYG{n}{Minimize}\PYG{p}{(}\PYG{n}{w}\PYG{p}{)}

\PYG{n}{constrs} \PYG{o}{=} \PYG{p}{[}\PYG{p}{]}

\PYG{n}{constrs} \PYG{o}{+}\PYG{o}{=} \PYG{p}{[}\PYG{n}{w} \PYG{o}{\PYGZgt{}}\PYG{o}{=} \PYG{n}{X}\PYG{p}{[}\PYG{l+m+mi}{0}\PYG{p}{,}\PYG{p}{:}\PYG{p}{]}\PYG{n+nd}{@t}\PYG{p}{]}
\PYG{n}{constrs} \PYG{o}{+}\PYG{o}{=} \PYG{p}{[}\PYG{n}{w}\PYG{o}{\PYGZgt{}}\PYG{o}{=} \PYG{n}{X}\PYG{p}{[}\PYG{l+m+mi}{1}\PYG{p}{,}\PYG{p}{:}\PYG{p}{]}\PYG{n+nd}{@t}\PYG{p}{]}
\PYG{n}{constrs} \PYG{o}{+}\PYG{o}{=} \PYG{p}{[}\PYG{n}{w} \PYG{o}{\PYGZgt{}}\PYG{o}{=} \PYG{n}{X}\PYG{p}{[}\PYG{l+m+mi}{2}\PYG{p}{,}\PYG{p}{:}\PYG{p}{]}\PYG{n+nd}{@t}\PYG{p}{]}

\PYG{k}{for} \PYG{n}{j} \PYG{o+ow}{in} \PYG{n+nb}{range}\PYG{p}{(}\PYG{l+m+mi}{15}\PYG{p}{)}\PYG{p}{:}
    \PYG{n}{constrs} \PYG{o}{+}\PYG{o}{=} \PYG{p}{[}\PYG{n}{X}\PYG{p}{[}\PYG{l+m+mi}{0}\PYG{p}{,}\PYG{n}{j}\PYG{p}{]}\PYG{o}{+}\PYG{n}{X}\PYG{p}{[}\PYG{l+m+mi}{1}\PYG{p}{,}\PYG{n}{j}\PYG{p}{]}\PYG{o}{+}\PYG{n}{X}\PYG{p}{[}\PYG{l+m+mi}{2}\PYG{p}{,}\PYG{n}{j}\PYG{p}{]} \PYG{o}{==} \PYG{l+m+mi}{1}\PYG{p}{]}

\PYG{k}{for} \PYG{n}{i} \PYG{o+ow}{in} \PYG{n+nb}{range}\PYG{p}{(}\PYG{l+m+mi}{3}\PYG{p}{)}\PYG{p}{:}
    \PYG{n}{constrs} \PYG{o}{+}\PYG{o}{=} \PYG{p}{[}\PYG{n}{X}\PYG{p}{[}\PYG{n}{i}\PYG{p}{,}\PYG{l+m+mi}{1}\PYG{p}{]}\PYG{o}{+}\PYG{n}{X}\PYG{p}{[}\PYG{n}{i}\PYG{p}{,}\PYG{l+m+mi}{4}\PYG{p}{]}\PYG{o}{+}\PYG{n}{X}\PYG{p}{[}\PYG{n}{i}\PYG{p}{,}\PYG{l+m+mi}{7}\PYG{p}{]} \PYG{o}{==} \PYG{l+m+mi}{1}\PYG{p}{]}
    \PYG{n}{constrs} \PYG{o}{+}\PYG{o}{=} \PYG{p}{[}\PYG{n}{X}\PYG{p}{[}\PYG{n}{i}\PYG{p}{,}\PYG{l+m+mi}{5}\PYG{p}{]}\PYG{o}{+}\PYG{n}{X}\PYG{p}{[}\PYG{n}{i}\PYG{p}{,}\PYG{l+m+mi}{8}\PYG{p}{]} \PYG{o}{\PYGZlt{}}\PYG{o}{=} \PYG{l+m+mi}{1}\PYG{p}{]}
    \PYG{n}{constrs} \PYG{o}{+}\PYG{o}{=} \PYG{p}{[}\PYG{n}{X}\PYG{p}{[}\PYG{n}{i}\PYG{p}{,}\PYG{l+m+mi}{6}\PYG{p}{]}\PYG{o}{+}\PYG{n}{X}\PYG{p}{[}\PYG{n}{i}\PYG{p}{,}\PYG{l+m+mi}{9}\PYG{p}{]} \PYG{o}{\PYGZlt{}}\PYG{o}{=} \PYG{l+m+mi}{1}\PYG{p}{]}
    \PYG{n}{constrs} \PYG{o}{+}\PYG{o}{=} \PYG{p}{[}\PYG{n}{X}\PYG{p}{[}\PYG{n}{i}\PYG{p}{,}\PYG{l+m+mi}{10}\PYG{p}{]}\PYG{o}{+}\PYG{n}{X}\PYG{p}{[}\PYG{n}{i}\PYG{p}{,}\PYG{l+m+mi}{14}\PYG{p}{]} \PYG{o}{\PYGZlt{}}\PYG{o}{=} \PYG{l+m+mi}{1}\PYG{p}{]}

\PYG{n}{prob} \PYG{o}{=} \PYG{n}{cp}\PYG{o}{.}\PYG{n}{Problem}\PYG{p}{(}\PYG{n}{obj}\PYG{p}{,} \PYG{n}{constrs}\PYG{p}{)}

\PYG{n}{prob}\PYG{o}{.}\PYG{n}{solve}\PYG{p}{(}\PYG{n}{solver}\PYG{o}{=}\PYG{n}{cp}\PYG{o}{.}\PYG{n}{GUROBI}\PYG{p}{)}
\end{sphinxVerbatim}


\subsection{Facility Location Problem}
\label{\detokenize{operationsResearch/IP:facility-location-problem}}
\sphinxAtStartPar
A city is divided into \(n\) districts. The time (in minutes) it takes an ambulance to travel from District \(i\) to
District \(j\) is denoted as \(d_{ij}\). The population of District \(i\) (in thousands) is \(p_{i}\).
An example is shown in Table 2 and Table 3.  The distances between districts are shown in Table 2, and the population
information is shown in Table 3.  In this instance, we have \(n = 8\) districts. We may see that, e.g., it takes 5
minutes to travel from District 2 to District 3, and there are 40,000 citizens.


\begin{savenotes}\sphinxattablestart
\centering
\sphinxcapstartof{table}
\sphinxthecaptionisattop
\sphinxcaption{Table 2: distances}\label{\detokenize{operationsResearch/IP:id2}}
\sphinxaftertopcaption
\begin{tabulary}{\linewidth}[t]{|T|T|T|T|T|T|T|T|T|}
\hline
\sphinxstyletheadfamily 
\sphinxAtStartPar
District
&\sphinxstyletheadfamily 
\sphinxAtStartPar
1
&\sphinxstyletheadfamily 
\sphinxAtStartPar
2
&\sphinxstyletheadfamily 
\sphinxAtStartPar
3
&\sphinxstyletheadfamily 
\sphinxAtStartPar
4
&\sphinxstyletheadfamily 
\sphinxAtStartPar
5
&\sphinxstyletheadfamily 
\sphinxAtStartPar
6
&\sphinxstyletheadfamily 
\sphinxAtStartPar
7
&\sphinxstyletheadfamily 
\sphinxAtStartPar
8
\\
\hline
\sphinxAtStartPar
1
&
\sphinxAtStartPar
0
&
\sphinxAtStartPar
3
&
\sphinxAtStartPar
4
&
\sphinxAtStartPar
6
&
\sphinxAtStartPar
8
&
\sphinxAtStartPar
9
&
\sphinxAtStartPar
8
&
\sphinxAtStartPar
10
\\
\hline
\sphinxAtStartPar
2
&
\sphinxAtStartPar
3
&
\sphinxAtStartPar
0
&
\sphinxAtStartPar
5
&
\sphinxAtStartPar
4
&
\sphinxAtStartPar
8
&
\sphinxAtStartPar
6
&
\sphinxAtStartPar
12
&
\sphinxAtStartPar
9
\\
\hline
\sphinxAtStartPar
3
&
\sphinxAtStartPar
4
&
\sphinxAtStartPar
5
&
\sphinxAtStartPar
0
&
\sphinxAtStartPar
2
&
\sphinxAtStartPar
2
&
\sphinxAtStartPar
3
&
\sphinxAtStartPar
5
&
\sphinxAtStartPar
7
\\
\hline
\sphinxAtStartPar
4
&
\sphinxAtStartPar
6
&
\sphinxAtStartPar
4
&
\sphinxAtStartPar
2
&
\sphinxAtStartPar
0
&
\sphinxAtStartPar
3
&
\sphinxAtStartPar
2
&
\sphinxAtStartPar
5
&
\sphinxAtStartPar
4
\\
\hline
\sphinxAtStartPar
5
&
\sphinxAtStartPar
8
&
\sphinxAtStartPar
8
&
\sphinxAtStartPar
2
&
\sphinxAtStartPar
3
&
\sphinxAtStartPar
0
&
\sphinxAtStartPar
2
&
\sphinxAtStartPar
2
&
\sphinxAtStartPar
4
\\
\hline
\sphinxAtStartPar
6
&
\sphinxAtStartPar
9
&
\sphinxAtStartPar
6
&
\sphinxAtStartPar
3
&
\sphinxAtStartPar
2
&
\sphinxAtStartPar
2
&
\sphinxAtStartPar
0
&
\sphinxAtStartPar
3
&
\sphinxAtStartPar
2
\\
\hline
\sphinxAtStartPar
7
&
\sphinxAtStartPar
8
&
\sphinxAtStartPar
12
&
\sphinxAtStartPar
5
&
\sphinxAtStartPar
5
&
\sphinxAtStartPar
2
&
\sphinxAtStartPar
3
&
\sphinxAtStartPar
0
&
\sphinxAtStartPar
2
\\
\hline
\sphinxAtStartPar
8
&
\sphinxAtStartPar
10
&
\sphinxAtStartPar
9
&
\sphinxAtStartPar
7
&
\sphinxAtStartPar
4
&
\sphinxAtStartPar
4
&
\sphinxAtStartPar
2
&
\sphinxAtStartPar
2
&
\sphinxAtStartPar
0
\\
\hline
\end{tabulary}
\par
\sphinxattableend\end{savenotes}


\begin{savenotes}\sphinxattablestart
\centering
\sphinxcapstartof{table}
\sphinxthecaptionisattop
\sphinxcaption{Table 3: District Population}\label{\detokenize{operationsResearch/IP:id3}}
\sphinxaftertopcaption
\begin{tabulary}{\linewidth}[t]{|T|T|}
\hline
\sphinxstyletheadfamily 
\sphinxAtStartPar
District
&\sphinxstyletheadfamily 
\sphinxAtStartPar
Population
\\
\hline
\sphinxAtStartPar
1
&
\sphinxAtStartPar
40
\\
\hline
\sphinxAtStartPar
2
&
\sphinxAtStartPar
30
\\
\hline
\sphinxAtStartPar
3
&
\sphinxAtStartPar
35
\\
\hline
\sphinxAtStartPar
4
&
\sphinxAtStartPar
20
\\
\hline
\sphinxAtStartPar
5
&
\sphinxAtStartPar
15
\\
\hline
\sphinxAtStartPar
6
&
\sphinxAtStartPar
50
\\
\hline
\sphinxAtStartPar
7
&
\sphinxAtStartPar
45
\\
\hline
\sphinxAtStartPar
8
&
\sphinxAtStartPar
60
\\
\hline
\end{tabulary}
\par
\sphinxattableend\end{savenotes}

\sphinxAtStartPar
The city has mm ambulances and wants to locate them to mm of the districts. For each district, the population\sphinxhyphen{}weighted
firefighting time is defined as the product of the district population times the amount of time it takes for the closest
ambulance to travel to it. The decision maker aims to locate the mm ambulances to minimize the maximum population\sphinxhyphen{}weighted
firefighting time among all districts.

\sphinxAtStartPar
As an example, suppose that \(m = 2\), \(n = 8\), \(d_{i,j}\) and \(p_{i}\) are provided in Table 2,
and the two ambulances are located in District 1 and 8. We then know that for Districts 1, 2, and 3 the closest ambulance
is in District 1 and for the remaining five districts the closet ambulance is in District 8. The firefighting time for the
eight districts are thus 0, 3, 4, 4, 4, 2, 2, and 0 minutes, respectively. The population\sphinxhyphen{}weighted firefighting times may
then be calculated as 0, 90, 140, 80, 60, 100, 90, and 0. The maximum among the eight districts is therefore 140.

\sphinxAtStartPar
For this problem, formulate an integer program that can minimize the maximum population\sphinxhyphen{}weighted firefighting time among all
districts. Then write a program to invoke a solver (e.g., write a Python program to invoke Gurobi Optimizer) to solve the
above instance and find an optimal solution for each problem. Write down the minimized maximum population\sphinxhyphen{}weighted
firefighting times among all districts of the two districts that ambulances should be located in (i.e., the objective value
of an optimal solution).
\begin{equation*}
\begin{split}\min_{X_{i,j}, w} w\end{split}
\end{equation*}
\sphinxAtStartPar
subject to:
\begin{equation*}
\begin{split}& \sum_{j \in \mathcal{N}} x_{j} = m \\
& Y_{i,j} \leq x_{j}, \forall i,j \in \mathcal{N} \\
& \sum_{j \in \mathcal{N}} Y_{i,j} = 1 , \forall i \in \mathcal{N} \\
& w \geq \sum_{j \in \mathcal{N}} d_{i,j} p_{i} Y_{i,j} , \forall i \in \mathcal{N} \\
& x_{i}, Y_{i,j} \in \{ 0,1 \}, \forall i,j \in \mathcal{N} \\
& w \geq 0\end{split}
\end{equation*}
\sphinxAtStartPar
where Decision variable \(Y_{i,j} = \left\{\begin{matrix}
1 & \text{if for District } i \text{ the cloest ambulance is located in District } j\\
0 & \text{otherwise}
\end{matrix}\right.\)

\sphinxAtStartPar
Decision variable \(x_{j} = \left\{\begin{matrix}
1 & \text{if an ambulance is located in District } j\\
0 & \text{otherwise}
\end{matrix}\right.\)

\sphinxAtStartPar
Decision variable \(w\) is a trivial positive value.

\begin{sphinxVerbatim}[commandchars=\\\{\}]
\PYG{k+kn}{import} \PYG{n+nn}{numpy} \PYG{k}{as} \PYG{n+nn}{np}
\PYG{k+kn}{import} \PYG{n+nn}{cvxpy} \PYG{k}{as} \PYG{n+nn}{cp}

\PYG{n}{m} \PYG{o}{=} \PYG{l+m+mi}{2}
\PYG{n}{n} \PYG{o}{=} \PYG{l+m+mi}{8}

\PYG{n}{d} \PYG{o}{=} \PYG{n}{np}\PYG{o}{.}\PYG{n}{array}\PYG{p}{(}\PYG{p}{[}\PYG{p}{[}\PYG{l+m+mi}{0}\PYG{p}{,}\PYG{l+m+mi}{3}\PYG{p}{,}\PYG{l+m+mi}{4}\PYG{p}{,}\PYG{l+m+mi}{6}\PYG{p}{,}\PYG{l+m+mi}{8}\PYG{p}{,}\PYG{l+m+mi}{9}\PYG{p}{,}\PYG{l+m+mi}{8}\PYG{p}{,}\PYG{l+m+mi}{10}\PYG{p}{]}\PYG{p}{,}
            \PYG{p}{[}\PYG{l+m+mi}{3}\PYG{p}{,}\PYG{l+m+mi}{0}\PYG{p}{,}\PYG{l+m+mi}{5}\PYG{p}{,}\PYG{l+m+mi}{4}\PYG{p}{,}\PYG{l+m+mi}{8}\PYG{p}{,}\PYG{l+m+mi}{6}\PYG{p}{,}\PYG{l+m+mi}{12}\PYG{p}{,}\PYG{l+m+mi}{9}\PYG{p}{]}\PYG{p}{,}
            \PYG{p}{[}\PYG{l+m+mi}{4}\PYG{p}{,}\PYG{l+m+mi}{5}\PYG{p}{,}\PYG{l+m+mi}{0}\PYG{p}{,}\PYG{l+m+mi}{2}\PYG{p}{,}\PYG{l+m+mi}{2}\PYG{p}{,}\PYG{l+m+mi}{3}\PYG{p}{,}\PYG{l+m+mi}{5}\PYG{p}{,}\PYG{l+m+mi}{7}\PYG{p}{]}\PYG{p}{,}
            \PYG{p}{[}\PYG{l+m+mi}{6}\PYG{p}{,}\PYG{l+m+mi}{4}\PYG{p}{,}\PYG{l+m+mi}{2}\PYG{p}{,}\PYG{l+m+mi}{0}\PYG{p}{,}\PYG{l+m+mi}{3}\PYG{p}{,}\PYG{l+m+mi}{2}\PYG{p}{,}\PYG{l+m+mi}{5}\PYG{p}{,}\PYG{l+m+mi}{4}\PYG{p}{]}\PYG{p}{,}
            \PYG{p}{[}\PYG{l+m+mi}{8}\PYG{p}{,}\PYG{l+m+mi}{8}\PYG{p}{,}\PYG{l+m+mi}{2}\PYG{p}{,}\PYG{l+m+mi}{3}\PYG{p}{,}\PYG{l+m+mi}{0}\PYG{p}{,}\PYG{l+m+mi}{2}\PYG{p}{,}\PYG{l+m+mi}{2}\PYG{p}{,}\PYG{l+m+mi}{4}\PYG{p}{]}\PYG{p}{,}
            \PYG{p}{[}\PYG{l+m+mi}{9}\PYG{p}{,}\PYG{l+m+mi}{6}\PYG{p}{,}\PYG{l+m+mi}{3}\PYG{p}{,}\PYG{l+m+mi}{2}\PYG{p}{,}\PYG{l+m+mi}{2}\PYG{p}{,}\PYG{l+m+mi}{0}\PYG{p}{,}\PYG{l+m+mi}{3}\PYG{p}{,}\PYG{l+m+mi}{2}\PYG{p}{]}\PYG{p}{,}
            \PYG{p}{[}\PYG{l+m+mi}{8}\PYG{p}{,}\PYG{l+m+mi}{12}\PYG{p}{,}\PYG{l+m+mi}{5}\PYG{p}{,}\PYG{l+m+mi}{5}\PYG{p}{,}\PYG{l+m+mi}{2}\PYG{p}{,}\PYG{l+m+mi}{3}\PYG{p}{,}\PYG{l+m+mi}{0}\PYG{p}{,}\PYG{l+m+mi}{2}\PYG{p}{]}\PYG{p}{,}
            \PYG{p}{[}\PYG{l+m+mi}{10}\PYG{p}{,}\PYG{l+m+mi}{9}\PYG{p}{,}\PYG{l+m+mi}{7}\PYG{p}{,}\PYG{l+m+mi}{4}\PYG{p}{,}\PYG{l+m+mi}{4}\PYG{p}{,}\PYG{l+m+mi}{2}\PYG{p}{,}\PYG{l+m+mi}{2}\PYG{p}{,}\PYG{l+m+mi}{0}\PYG{p}{]}\PYG{p}{]}\PYG{p}{)}

\PYG{n}{p} \PYG{o}{=} \PYG{n}{np}\PYG{o}{.}\PYG{n}{array}\PYG{p}{(}\PYG{p}{[}\PYG{p}{[}\PYG{l+m+mi}{40}\PYG{p}{,}\PYG{l+m+mi}{30}\PYG{p}{,}\PYG{l+m+mi}{35}\PYG{p}{,}\PYG{l+m+mi}{20}\PYG{p}{,}\PYG{l+m+mi}{15}\PYG{p}{,}\PYG{l+m+mi}{50}\PYG{p}{,}\PYG{l+m+mi}{45}\PYG{p}{,}\PYG{l+m+mi}{60}\PYG{p}{]}\PYG{p}{]}\PYG{p}{)}\PYG{o}{.}\PYG{n}{T}

\PYG{n}{x} \PYG{o}{=} \PYG{n}{cp}\PYG{o}{.}\PYG{n}{Variable}\PYG{p}{(}\PYG{p}{(}\PYG{n}{n}\PYG{p}{,}\PYG{l+m+mi}{1}\PYG{p}{)}\PYG{p}{,} \PYG{n}{boolean}\PYG{o}{=}\PYG{k+kc}{True}\PYG{p}{)}
\PYG{n}{Y} \PYG{o}{=} \PYG{n}{cp}\PYG{o}{.}\PYG{n}{Variable}\PYG{p}{(}\PYG{p}{(}\PYG{n}{n}\PYG{p}{,}\PYG{n}{n}\PYG{p}{)}\PYG{p}{,} \PYG{n}{boolean}\PYG{o}{=}\PYG{k+kc}{True}\PYG{p}{)}
\PYG{n}{w} \PYG{o}{=} \PYG{n}{cp}\PYG{o}{.}\PYG{n}{Variable}\PYG{p}{(}\PYG{p}{)}

\PYG{n}{obj} \PYG{o}{=} \PYG{n}{cp}\PYG{o}{.}\PYG{n}{Minimize}\PYG{p}{(}\PYG{n}{w}\PYG{p}{)}

\PYG{n}{constrs} \PYG{o}{=} \PYG{p}{[}\PYG{n}{cp}\PYG{o}{.}\PYG{n}{sum}\PYG{p}{(}\PYG{n}{x}\PYG{p}{)} \PYG{o}{==} \PYG{n}{m}\PYG{p}{]}

\PYG{k}{for} \PYG{n}{i} \PYG{o+ow}{in} \PYG{n+nb}{range}\PYG{p}{(}\PYG{n}{n}\PYG{p}{)}\PYG{p}{:}
    \PYG{n}{constrs} \PYG{o}{+}\PYG{o}{=} \PYG{p}{[}\PYG{n}{cp}\PYG{o}{.}\PYG{n}{reshape}\PYG{p}{(}\PYG{n}{Y}\PYG{p}{[}\PYG{n}{i}\PYG{p}{,}\PYG{p}{:}\PYG{p}{]}\PYG{p}{,}\PYG{p}{(}\PYG{l+m+mi}{1}\PYG{p}{,}\PYG{n}{n}\PYG{p}{)}\PYG{p}{)} \PYG{o}{\PYGZlt{}}\PYG{o}{=} \PYG{n}{x}\PYG{o}{.}\PYG{n}{T}\PYG{p}{]}

\PYG{k}{for} \PYG{n}{i} \PYG{o+ow}{in} \PYG{n+nb}{range}\PYG{p}{(}\PYG{n}{n}\PYG{p}{)}\PYG{p}{:}
    \PYG{n}{constrs} \PYG{o}{+}\PYG{o}{=} \PYG{p}{[}\PYG{n}{cp}\PYG{o}{.}\PYG{n}{sum}\PYG{p}{(}\PYG{n}{Y}\PYG{p}{[}\PYG{n}{i}\PYG{p}{,}\PYG{p}{:}\PYG{p}{]}\PYG{p}{)} \PYG{o}{==} \PYG{l+m+mi}{1}\PYG{p}{]}

\PYG{k}{for} \PYG{n}{i} \PYG{o+ow}{in} \PYG{n+nb}{range}\PYG{p}{(}\PYG{n}{n}\PYG{p}{)}\PYG{p}{:}
    \PYG{n}{constrs} \PYG{o}{+}\PYG{o}{=} \PYG{p}{[}\PYG{n}{w} \PYG{o}{\PYGZgt{}}\PYG{o}{=} \PYG{n}{d}\PYG{p}{[}\PYG{n}{i}\PYG{p}{,}\PYG{n}{j}\PYG{p}{]}\PYG{o}{*}\PYG{n}{p}\PYG{p}{[}\PYG{n}{i}\PYG{p}{]}\PYG{o}{*}\PYG{n}{Y}\PYG{p}{[}\PYG{n}{i}\PYG{p}{,}\PYG{n}{j}\PYG{p}{]} \PYG{k}{for} \PYG{n}{j} \PYG{o+ow}{in} \PYG{n+nb}{range}\PYG{p}{(}\PYG{n}{n}\PYG{p}{)}\PYG{p}{]}


\PYG{n}{prob} \PYG{o}{=} \PYG{n}{cp}\PYG{o}{.}\PYG{n}{Problem}\PYG{p}{(}\PYG{n}{obj}\PYG{p}{,} \PYG{n}{constrs}\PYG{p}{)}

\PYG{n}{prob}\PYG{o}{.}\PYG{n}{solve}\PYG{p}{(}\PYG{n}{solver}\PYG{o}{=}\PYG{n}{cp}\PYG{o}{.}\PYG{n}{GUROBI}\PYG{p}{)}
\end{sphinxVerbatim}


\chapter{Nonlinear Programming}
\label{\detokenize{operationsResearch/NLP:nonlinear-programming}}\label{\detokenize{operationsResearch/NLP::doc}}
\begin{sphinxShadowBox}
\sphinxstyletopictitle{Contents}
\begin{itemize}
\item {} 
\sphinxAtStartPar
\phantomsection\label{\detokenize{operationsResearch/NLP:id1}}{\hyperref[\detokenize{operationsResearch/NLP:nonlinear-programming}]{\sphinxcrossref{Nonlinear Programming}}}
\begin{itemize}
\item {} 
\sphinxAtStartPar
\phantomsection\label{\detokenize{operationsResearch/NLP:id2}}{\hyperref[\detokenize{operationsResearch/NLP:nlp-model}]{\sphinxcrossref{NLP Model}}}

\item {} 
\sphinxAtStartPar
\phantomsection\label{\detokenize{operationsResearch/NLP:id3}}{\hyperref[\detokenize{operationsResearch/NLP:solution-methods}]{\sphinxcrossref{Solution Methods}}}
\begin{itemize}
\item {} 
\sphinxAtStartPar
\phantomsection\label{\detokenize{operationsResearch/NLP:id4}}{\hyperref[\detokenize{operationsResearch/NLP:gradient-descent}]{\sphinxcrossref{Gradient Descent}}}

\item {} 
\sphinxAtStartPar
\phantomsection\label{\detokenize{operationsResearch/NLP:id5}}{\hyperref[\detokenize{operationsResearch/NLP:newton-s-method}]{\sphinxcrossref{Newton’s Method}}}

\end{itemize}

\item {} 
\sphinxAtStartPar
\phantomsection\label{\detokenize{operationsResearch/NLP:id6}}{\hyperref[\detokenize{operationsResearch/NLP:applications}]{\sphinxcrossref{Applications}}}

\end{itemize}

\end{itemize}
\end{sphinxShadowBox}


\section{NLP Model}
\label{\detokenize{operationsResearch/NLP:nlp-model}}

\section{Solution Methods}
\label{\detokenize{operationsResearch/NLP:solution-methods}}

\subsection{Gradient Descent}
\label{\detokenize{operationsResearch/NLP:gradient-descent}}

\subsection{Newton’s Method}
\label{\detokenize{operationsResearch/NLP:newton-s-method}}

\section{Applications}
\label{\detokenize{operationsResearch/NLP:applications}}

\chapter{Indices and tables}
\label{\detokenize{index:indices-and-tables}}\begin{itemize}
\item {} 
\sphinxAtStartPar
\DUrole{xref,std,std-ref}{genindex}

\item {} 
\sphinxAtStartPar
\DUrole{xref,std,std-ref}{modindex}

\item {} 
\sphinxAtStartPar
\DUrole{xref,std,std-ref}{search}

\end{itemize}



\renewcommand{\indexname}{Index}
\printindex
\end{document}